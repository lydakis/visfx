%!TEX root = ../main.tex



\section{Εισαγωγή στα Συστήματα Συστάσεων}

\lettrine[findent=2pt]{\fbox{\textbf{Σ}}}{ε} πολλά πεδία, καταναλωτές παρουσιάζονται με μια πληθώρα επιλογών και καλούνται να λάβουν κάποια απόφαση. Αυτή μπορεί να είναι ποια ταινία να δουν στο Netflix, ποιο βιβλίο να αγοράσουν από το Amazon, ή στην περίπτωσή μας ποιο ζεύγος συναλλάγματος να παίξουν σε μια πλατφόρμα Forex Trading. Τα Συστήματα Συστάσεων έρχονται να βοηθήσουν τους χρήστες τους δίνοντας συστάσεις βάση διαφόρων παραγόντων, όπως τα προηγούμενα ζεύγη που έπαιξαν, το ιστορικό αγορών τους ή τις αξιολογήσεις ταινιών που έχουν κάνει. 

\section{Τύποι Συστημάτων Συστάσεων}

\lettrine[findent=2pt]{\fbox{\textbf{Τ}}}{α} Συστήματα Συστάσεων μπορούν να χωριστούν σε δύο κατηγορίες ανάλογα με τη προσέγγιση που ακολουθούν· τα Βασισμένα σε Περιεχόμενο (Content Based) και τα Συνεργατικού Φιλτραρίσματος (Collaborative Filtering). Τα πρώτα σκοπεύουν στη δημιουργία ενός προφίλ για κάθε χρήστη ή κάθε προϊόν τα οποία χρησιμοποιούνται για να βρεθούν σχέσεις μεταξύ των χρηστών και προϊόντων. Τα προφίλ μπορούν να περιέχουν στατιστικά ή απαντήσεις από σχετικά ερωτηματολόγια και γενικότερα εξωτερικές πληροφορίες, που μερικές φορές είναι δύσκολο να συγκεντρωθούν. Ένα γνωστό Content Based σύστημα είναι το Music Genome Project, που χρησιμοποιείται από την υπηρεσία ιντερνετικού ραδιοφώνου Pandora.

Από την άλλη πλευρά τα συστήματα Συνεργατικού Φιλτραρίσματος (CF) βασίζονται μόνο σε προηγούμενες συναλλαγές ή αξιολογήσεις, αναλύοντας σχέσεις μεταξύ χρηστών και προϊόντων. Το μεγάλο πλεονέκτημα αυτών των συστημάτων είναι ότι είναι ανεξάρτητα του πεδίου που εφαρμόζονται αλλά παρόλα αυτά είναι ικανά μοντελοποιήσουν πλευρές των δεδομένων που είναι δύσκολο να φανούν χρησιμοποιώντας Content Based συστήματα. Ενώ γενικά τα CF συστήματα είναι πιο αποδοτικά, πάσχουν από το πρόβλημα της «κρύας εκκίνησης» (cold start), δε μπορούν με άλλα λόγια να ανταποκριθούν άμεσα σε νέους χρήστες ή νέα προϊόντα—χρειάζεται το σύστημα να εκπαιδευτεί ξανά στα νέα δεδομένα. Στη διπλωματική χρησιμοποιήθηκε τέτοιου τύπο RS.

Υπάρχουν δύο βασικές υλοποιήσεις Συνεργατικού Φιλτραρίσματος. Οι μέθοδοι γειτνίασης (neighborhood models) και τα μοντέλα λανθανουσών παραγόντων (latent factor models). Τα πρώτα εστιάζουν στο υπολογισμό των σχέσεων μεταξύ προϊόντων ή μεταξύ χρηστών. Η προσέγγιση βασισμένη στα προϊόντα δίνει την προτίμηση ενός χρήστη βάση αξιολογήσεων «γειτονικών» προϊόντων του ίδιου χρήστη. Γειτονικά προϊόντα είναι αυτά που τείνουν να παίρνουν παρόμοιες αξιολογήσεις όταν αξιολογούνται από τον ίδιο χρήστη. Ενώ αντίστοιχα η προσέγγιση βασισμένη στους χρήστες χρησιμοποιεί τις σχέσεις γειτνίασης μεταξύ των χρηστών. 

Από την άλλη τα μοντέλα λανθανουσών παραγόντων προσπαθούν να εξηγήσουν τις αξιολογήσεις των χρηστών και των προϊόντων υπολογίζοντας έναν αριθμό από παράγοντες οι οποίοι συνάγονται από μοτίβα μέσα στις αξιολογήσεις. Υπό μία έννοια, αυτοί οι παράγοντες είναι τα υπολογισμένα υποκατάστατα των στοιχείων ενός προφίλ μίας Content Based μεθόδου.

Στη δική μας υλοποίηση χρησιμοποιούμε CF με τη χρήση μοντέλων λανθανουσών παραγόντων με έμμεσες αξιολογήσεις. Αυτό σημαίνει ότι οι αξιολογήσεις, που καλούνται και «ψεύδο-αξιολογήσεις», δεν έχουν δοθεί άμεσα από τους χρήστες αλλά έχουν παραχθεί από δεδομένα των χρηστών όπως ο μέσος χρόνος συναλλαγής ή τα καθαρά κέρδη και ζημιές από τις συναλλαγές τους. 

Στη βιβλιογραφία δεν υπάρχει κάποια συγκεκριμένη μέθοδος συνδυασμού των χαρακτηριστικών για να παραχθούν οι ψευδο-αξιολογήσεις, καθώς ανάλογα με το πεδίο στο οποίο δουλεύουμε διαφορετικές μετρικές έχουν νόημα. Το γεγονός αυτό αφήνει πολλά περιθώρια στον σχεδιαστή του συστήματος CF να συνδυάσει τα δεδομένα που έχει και να παράγει τις ψευδο-αξιολογήσεις. 

\section{Συνεργατικό Φιλτράρισμα με χρήση Μοντέλου Λανθανουσών Παραγόντων με Έμμεσες Αξιολογήσεις
}

\lettrine[findent=2pt]{\fbox{\textbf{Τ}}}{α} μοντέλα λανθανουσών παραγόντων αντιστοιχούν τους χρήστες και τα προϊόντα σε ένα κοινό $f$-διάστατο χώρο λανθανουσών παραγόντων, έτσι ώστε οι σχέσεις μεταξύ χρηστών και προϊόντων να μοντελοποιούνται σαν το εσωτερικό τους γινόμενο σ’ αυτό τον χώρο. Κατ’ αυτόν τον τρόπο, κάθε προϊόν $i$ αντιστοιχίζεται με ένα διάνυσμα $x \in \mathbb{R}^f$ και κάθε χρήστης $u$ αντιστοιχίζεται με ένα διάνυσμα $y \in \mathbb{R}^f$. Δοθέντος ενός προϊόντος $i$, τα στοιχεία του $x_i$ μετράνε τον βαθμό στον οποίο το προϊόν κατέχει αυτούς τους παράγοντες, θετικούς ή αρνητικούς. Επίσης, δοθέντος ενός χρήστη $u$ τα στοιχεία του διανύσματος $y_u$ μετράνε το βαθμό ενδιαφέροντος που έχει ο χρήστης σε προϊόντα που έχουν υψηλές τιμές στους αντίστοιχους παράγοντες, πάλι θετικούς ή αρνητικούς. Το εσωτερικό γινόμενο, $x_i^Ty_u$, δείχνει την αλληλεπίδραση μεταξύ του χρήστη $u$ και του προϊόντος $i$—ουσιαστικά το συνολικό ενδιαφέρον του χρήστη στα χαρακτηριστικά του προϊόντος. Αυτό το γινόμενο προσεγγίζει την αξιολόγηση του χρήστη $u$ για το προϊόν $i$, η οποία συμβολίζεται $r_{ui}$, έτσι έχουμε:

\begin{equation}
	\hat{r}_{ui}=x_i^Ty_u\label{eq:1}
\end{equation}

Το κύριο πρόβλημα είναι ο υπολογισμός των διανυσμάτων παραγόντων $x_i$, $y_u$ για κάθε προϊον και κάθε χρήστη. Αφού το σύστημα συστάσεων ολοκληρώσει αυτόν τον υπολογισμό μπορεί εύκολα να εκτιμήσει τη αξιολόγηση που ένας χρήστης θα δώσει σε ένα προϊόν χρησιμοποιώντας την Εξίσωση \eqref{eq:1}. Για να «μάθει» το σύστημα τα διανύσματα παραγόντων, το σύστημα ελαχιστοποιεί το κανονικοποιημένο τετραγωνικό σφάλμα στο σύνολο το γνωστών αξιολογήσεων:

\begin{equation}
	\min_{x^*,y^*}\sum_{(u,i) \in \kappa} \left(r_{ui}-x_i^Ty_u \right)^2+\lambda\left(\|x_i\|^2+\|y_u\|^2\right)\label{eq:2}
\end{equation}

Όπου το $\kappa$ είναι το σύνολο των ζευγών $(u,i)$ όπου το $r_{ui}$ είναι γνωστό (σύνολο εκπαίδευσης). Το σύστημα μαθαίνει το μοντέλο προσεγγίζοντας τις προηγούμενες γνωστές αξιολογήσεις. Όμως ο σκοπός είναι να γενικοποιήσουμε αυτές τις προηγούμενες αξιολογήσεις έτσι ώστε να προβλέπουν μελλοντικές άγνωστες αξιολογήσεις. Γι’ αυτό, το σύστημα θα πρέπει να αποφεύγει την υπερπροσαρμογή (overfitting) των δεδομένων κανονικοποιώντας τους παραμέτρους που έμαθε. Η σταθερά $\lambda$ ρυθμίζει την έκταση της κανονικοποίησης, η οποία υπολογίζεται συνήθως μέσω διασταύρωσης (cross-validation).

Στην περίπτωσή μας όμως έχουμε έμμεσες και όχι άμεσες αξιολογήσεις· ο χρήστης δεν βάζει μία βαθμολογία σε κάθε συναλλαγή, όμως έχουμε μέτα-δεδομένα από τις συναλλαγές του χρήστη που μας δείχνουν την προτίμηση του χρήστη στο προϊόν (το νομισματικό ζεύγος). Γι’ αυτό το λόγο ορίζουμε ένα δυαδικό μέγεθος που ονομάζουμε προτίμηση  το οποίο δείχνει τη προτίμηση του χρήστη $u$ στο προϊόν $i$. Οι τιμές του $p_{ui}$ υπολογίζονται ψηφιοποιώντας τις τιμές του $r_{ui}$:

\begin{equation*}
	p_{ui}= \begin{cases}
		1 & r_{ui}>0 \\
		0 & r_{ui}=0
	\end{cases}
\end{equation*}

Με άλλα λόγια, αν ο χρήστης $u$ έχει συναλλαχθεί το αντικείμενο $i$ $\left(r_{ui}>0\right)$, τότε έχουμε μία ένδειξη ότι το $i$ αρέσει στον $u$ $\left(p_{ui}=1\right)$. Από την άλλη, αν ο $u$ δεν έχει ποτέ συναλλαχθεί το $i$, τότε πιστεύουμε ότι δεν υπάρχει προτίμηση $\left(p_{ui}=0\right)$. Όμως τα τι πιστεύουμε σχετίζονται με συχνά μεταβαλλόμενα επίπεδα εμπιστοσύνης. Πρώτα πρώτα, από τη φύση τους οι μηδενικές τιμές του  σχετίζονται με χαμηλή εμπιστοσύνη, γιατί το γεγονός ότι ο χρήστης συναλλάχθηκε αυτό το προϊόν μπορεί να εξαρτάται από άλλους παραμέτρους πέρα από ότι δεν του άρεσε. Για παράδειγμα μπορεί να μην γνώριζε την ύπαρξη του προϊόντος ή η τιμή του να μην ήταν ευνοϊκή εκείνη τη χρονική περίοδο. Κατ’ αυτόν τον τρόπο έχουμε διάφορα επίπεδα εμπιστοσύνης ακόμα κι ανάμεσα σε προϊόντα στα οποία έχουμε ένδειξη προτίμησης από τον χρήστη. Γενικά, όσο το $p_{ui}$ μεγαλώνει, έχουμε ισχυρότερη ένδειξη ότι το προϊόν πραγματικά αρέσει στον χρήστη. Γι’ αυτό τον λόγο εισάγουμε ένα νέο σετ από μεταβλητές $c_{ui}$, όπου μετράμε την εμπιστοσύνη στην προτίμηση $p_{ui}$. Μία πιθανή επιλογή για το $c_{ui}$ είναι:

\begin{equation*}
	c_{ui}=1+\alpha r_{ui}
\end{equation*}

Μ’ αυτόν τον τρόπο έχουμε μία ελάχιστη εμπιστοσύνη στο $p_{ui}$ για κάθε ζεύγος χρήστη-προϊόντος, αλλά όσο παρατηρούμε περισσότερες αποδείξεις για θετική προτίμηση, η εμπιστοσύνη μας στο $p_{ui}$ αυξάνεται αντίστοιχα. Ο βαθμός με τον οποίο αυξάνεται η εμπιστοσύνη ελέγχεται από την σταθερά $\alpha$. 

Έτσι, για να λάβουμε υπόψιν τις έμμεσες αξιολογήσεις βάση τα παραπάνω η Εξίσωση \eqref{eq:2} μετασχηματίζεται ως εξής:

\begin{equation}
	\min_{x^*,y^*}\sum_{u,i} c_{ui}\left(p_{ui}-x_i^Ty_u \right)^2+\lambda\left(\sum_i\|x_i\|^2+\sum_u\|y_u\|^2\right)\label{eq:3}
\end{equation}

Κανείς μπορεί να παρατηρήσει ότι η συνάρτηση κόστους περιέχει $m \cdot n$ όρους, όπου το $m$ είναι ο αριθμός των χρηστών και $n$ ο αριθμός των προϊόντων. Για τυπικά σετ δεδομένων το $m \cdot n$ μπορεί να φτάσει μερικά εκατομμύρια. Ο μεγάλος αριθμός από όρους εμποδίζει την εφαρμογή άμεσων τεχνικών βελτιστοποίησης όπως η στοχαστική γραμμική παλινδρόμηση, η οποία χρησιμοποιείται ευρέως σε σετ δεδομένων με άμεσες αξιολογήσεις. 

Γι’ αυτό το λόγο για τη μάθηση των παραμέτρων επιλέχθηκε η μέθοδος των Εναλλασσόμενων Ελαχίστων Τετραγώνων (Alternating Least Squares – ALS). Επειδή τόσο το $x_i$ όσο και το $y_u$ είναι άγνωστοι η Εξίσωση \eqref{eq:3} δεν είναι κυρτή. Παρόλα αυτά, αν κρατήσουμε έναν από τους δύο αγνώστους σταθερό το πρόβλημα ελαχιστοποίησης γίνεται τετραγωνικό και υπάρχει βέλτιστη λύση. Έτσι, η τεχνική ALS κρατάει εναλλάξ μία τα $x_i$ και μία τα $y_u$ σταθερά. Όταν τα $y_u$ είναι σταθερά, το σύστημα ξαναυπολογίζει τα $x_i$ λύνοντας ένα πρόβλημα ελαχίστων τετραγώνων και αντίστοιχα πράττει όταν τα $x_i$ είναι σταθερά. Αυτό εξασφαλίζει ότι σε κάθε βήμα μειώνεται η Εξίσωση \eqref{eq:3} μέχρι να επιτευχθεί σύγκλιση.
